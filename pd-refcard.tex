%% pd-refcard.tex : Pure Data Reference Card
%% 2010/12/13 by Karim Barkati
%% karim.barkati@gmail.com
%% 2018/11/08 by Niccolò Granieri
%% granieriniccolo@gmail.com
%% Copyright (c) 2010 GFDL 1.3

\documentclass[a4paper, landscape, 9pt]{extarticle}
\usepackage[utf8]{inputenc}
\usepackage[T1]{fontenc}
\usepackage{tabularx}
\usepackage{multicol}
\usepackage{geometry}
\usepackage{hyperref}
\usepackage{url}
\urlstyle{sf}
\thispagestyle{empty}
\geometry{hmargin=0.7cm, vmargin=0.5cm}
\hypersetup{
  colorlinks = true,
  urlcolor = black
}

\newcommand{\audio}{\raisebox{-0.7mm}[0mm][0.1mm]{\~\}}
\newcommand{\email}[1]{\href{mailto:#1}{\textsf{#1}}}
\newcommand{\refcardtitle}[1]{
  \begin{center}
    \textbf{\small{#1}}
  \end{center}
}


\begin{document}

\begin{multicols}{3}

  \begin{center}
    \Large{\textbf{Pure Data Reference Card}} \\
    \small{K. \textsc{Barkati} \& N. \textsc{Granieri} -- \today} 
  \end{center}

  \footnotesize

  \refcardtitle{Modes}
  \begin{tabularx}{9.1cm}{X}
    \texttt{ctrl-e} (or \texttt{cmd-e}) toggle between \emph{run} mode (performance) and \emph{edit} mode (programming); this affects how mouse clicks affect the patch.
  \end{tabularx}

  \refcardtitle{General}
  \begin{tabularx}{9cm}{>{\tt}l X}
    bang & output a bang message \\
    float & store and recall a number  \\
    symbol & store and recall a symbol \\
    int & store and recall an integer  \\
    send & send a message to a named object  \\
    receive & catch "sent" messages  \\
    select & test for matching numbers or symbols  \\
    route & route messages according to first element  \\
    pack & make compound messages  \\
    unpack & get elements of compound messages  \\
    trigger & sequence and convert messages  \\
    spigot & interruptible message connection  \\
    moses & part a numeric stream  \\
    until & looping mechanism  \\
    print & print out messages  \\
    makefilename & format a symbol with a variable field  \\
    change & remove repeated numbers from a stream  \\
    swap & swap two numbers  \\
    value & shared numeric value \\
    list & manipulate lists \\
  \end{tabularx}

  \refcardtitle{Time}
  \begin{tabularx}{9cm}{>{\tt}l X}
    delay & send a message after a time delay \\
    metro & send a message periodically \\
    line & send a series of linearly stepped numbers \\
    timer & measure time intervals \\
    cputime & measure CPU time \\
    realtime & measure real time \\
    pipe & dynamically growable delay line for messages \\
  \end{tabularx}

  \refcardtitle{Math}
  \begin{tabularx}{9cm}{>{\tt}X X}
  	expr & C-style expressions \\
  	+ - * / pow  & arithmetic \\
    == != > < >= <= & relational tests \\
    \& \&\& | || \% << >> & bit twiddling \\
    mtof ftom powtodb rmstodb dbtopow dbtorms & convert acoustical units \\
    mod div sin cos tan atan atan2 sqrt log exp abs & higher math \\
    random & lower math \\ 
    max min & greater or lesser of 2 numbers \\
    clip & force a number into a range \\
    wrap & wrap a number to a range [0, 1)
  \end{tabularx}
  
  
  \refcardtitle{I/O via MIDI, OSC and FUDI}
  \begin{tabularx}{9cm}{>{\tt}X X}
    notein ctlin pgmin bendin touchin polytouchin midiin sysexin midirealtimein & MIDI input \\
    noteout ctlout pgmout bendout touchout polytouchout midiout & MIDI output \\
  \end{tabularx}
  \begin{tabularx}{9cm}{>{\tt}X X}
    makenote & schedule delayed "note off" message for a note-on \\
    stripnote & strip note-off messages \\
    oscparse oscformat & OSC messages to and from Pd lists \\
    fudiparse fudiformat & FUDI messages to and from Pd lists \\
  \end{tabularx}

\columnbreak
  \refcardtitle{Arrays / Tables}
  \begin{tabularx}{9cm}{>{\tt}l X}
    tabread & read a number from a table \\
    tabread4 & read a number from a table with 4 point interpolation \\
    tabwrite & write a number to a table \\
    soundfiler & read and write tables to soundfiles \\
    table & create a named table \\
    array & general array creation and manipulation \\
  \end{tabularx}

  \refcardtitle{Misc}
  \begin{tabularx}{9cm}{>{\tt}l X}
    loadbang & bang on load \\
    declare & set search path and/or load libraries \\
    savestate & mechanism for saving state of an abstraction \\
    netsend & send messages over the internet \\
    netreceive & receive them \\
    qlist & message sequencer \\
    textfile & file to message converter \\
    text & general text handling \\
    openpanel & "Open" dialog \\
    savepanel & "Save as" dialog \\
    bag & set of numbers \\
    poly & polyphonic voice allocation \\
    key, keyup & numeric key values from keyboard \\
    keyname & symbolic key name \\
  \end{tabularx}

  \refcardtitle{Audio Math}
  \begin{tabularx}{9cm}{>{\tt}l X}
  	expr\audio fexpr\audio & C-style expressions \\
    +\audio -\audio *\audio /\audio & arithmetic on audio signals \\
    max\audio min\audio & maximum or minimum of 2 inputs \\
    clip\audio & constrict signal to lie between two bounds \\
    sqrt\audio & approximate (16-bit) square root \\
    rsqrt\audio & reciprocal square root \\
    q8\_rsqrt\audio q8\_sqrt\audio & fast, cheap 8 bits versions \\
    wrap\audio & wraparound (fractional part) \\
    fft\audio & complex forward discrete Fourier transform \\
    ifft\audio & complex inverse discrete Fourier transform \\
    rfft\audio & real forward discrete Fourier transform \\
    rifft\audio & real inverse discrete Fourier transform \\
    pow\audio log\audio exp\audio abs\audio & math\\
    framp\audio & output a ramp for each block \\
  \end{tabularx}
  \begin{tabularx}{9cm}{>{\tt}l X}
    mtof\audio ftom\audio rmstodb\audio dbtorms\audio & acoustic conversions \\
  \end{tabularx}

  \refcardtitle{General Audio Manipulation}
  \begin{tabularx}{9cm}{>{\tt}l X}
    dac\audio & audio output \\
    adc\audio & audio input \\ 
    sig\audio & convert numbers to audio signals \\ 
    line\audio & generate audio ramps \\ 
    vline\audio & deluxe \texttt{line\~} \\ 
    threshold\audio & detect signal thresholds \\
    snapshot\audio & sample a signal (convert it back to a number) \\
    vsnapshot\audio & deluxe \texttt{snapshot\~} \\
    bang\audio & send a bang message after each DSP block \\
    samplerate\audio & get the sample rate \\
    send\audio & nonlocal signal connection with fanout \\
    receive\audio & get signal from \texttt{send\~} \\
    throw\audio & add to a summing bus \\
    catch\audio & define and read a summing bus \\
    readsf\audio & soundfile playback from disk \\
    writesf\audio & record sound to disk \\
  \end{tabularx}
  
  \refcardtitle{Audio Oscillators and Tables}
  \begin{tabularx}{9cm}{>{\tt}l X}
    phasor\audio & sawtooth oscillator \\
    cos\audio & cosine \\
    osc\audio & cosine oscillator \\
    tabwrite\audio & write to a table \\
    tabplay\audio & play back from a table (non-transposing) \\
    tabread\audio & non-interpolating table read \\
    tabread4\audio & four-point interpolating table read \\
    tabosc4\audio & wavetable oscillator \\ 
    tabsend\audio & write one block continuously to a table \\
    tabreceive\audio & read one block continuously from a table \\
  \end{tabularx}
  
  \refcardtitle{Audio Filters}
  \begin{tabularx}{9cm}{>{\tt}l X}
    vcf\audio & voltage controlled filter \\
    noise\audio &  white noise generator \\
    env\audio & envelope follower (RMS amplitude in dB) \\
    hip\audio &  high pass filter \\
    lop\audio &  low pass filter \\ 
    bp\audio &  band pass filter \\
    biquad\audio & raw filter (2 poles and 2 zeros)\\ 
    samphold\audio & sample and hold unit \\
    print\audio & print out one or more "blocks" \\
    rpole\audio & raw real-valued one-pole filter \\
    rzero\audio & raw real-valued one-zero filter \\
    rzero\_rev\audio & time-reversed \texttt{rzero\~} \\
  \end{tabularx}
  \texttt{cpole\~, czero\~, czero\_rev\~} corresponding complex-valued filters

  \refcardtitle{Audio Delay}
  \begin{tabularx}{9cm}{>{\tt}l X}
    delwrite\audio & write to a delay line \\
    delread\audio & read from a delay line \\
    delread4\audio vd\audio & read with a time-varying delay time \\
  \end{tabularx}
  
  \refcardtitle{Subwindows}
  \begin{tabularx}{9cm}{>{\tt}l X}
    pd & define a subwindow \\
    inlet & add an inlet to a pd \\
    outlet & add an outlet to a pd \\
    inlet\audio, outlet\audio & signal versions of \texttt{inlet} and \texttt{outlet} \\
    clone & make copies of a subpatch \\
    block\audio switch & specify block size and overlap, or, if invoked as "switch", also switch subpatches on and off \\
  \end{tabularx}

  \refcardtitle{Data Templates}
  \begin{tabularx}{9cm}{>{\tt}l X}
    struct & define a data structure \\
    drawcurve, filledcurve & draw a curve \\
    drawpolygon, filledpolygon & draw a polygon \\
    drawtext drawsymbol & draw text \\
    plot & plot an array field \\
    drawnumber & print a numeric value \\
    \end{tabularx}

  \refcardtitle{Accessing Data}
  \begin{tabularx}{9cm}{>{\tt}l X}
    pointer & point to an object belonging to a template \\
    get & get numeric fields \\
    set & change numeric fields \\
    element & get an array element \\
    getsize & get the size of an array \\
    setsize & change the size of an array \\
    append & add an element to a list \\
    scalar & create a single scalar \\
  \end{tabularx}
  
  \refcardtitle{Extra (abstractions and externals in pd/extra)}
  \begin{tabularx}{9cm}{>{\tt}l X}
  	sigmund\audio & pitch tracker \\
  	bonk\audio & attack detector \\
  	choice & best match of list to templates \\
  	hilbert\audio complex-mod\audio & phase quadrature / frequency shifting \\
  	loop\audio & phasor\audio with S/H on its frequency input\\
  	lrshift\audio & left and right shift (useful with FFT objects) \\
  	pd\audio stout & run another copy of Pd (for multiprocessing) \\ 
  	rev1\audio rev2\audio rev3\audio & reverberators \\
  	bob\audio & Moog resonant filter model \\
  	
  \end{tabularx}
  
  \medskip{} 
  \noindent{}
  \begin{tabularx}{9cm}{X}
    \tiny{Copyright \copyright{2018} Karim \textsc{Barkati} <\email{karim.barkati@gmail.com}> \& Niccolò \textsc{Granieri} <\email{granieriniccolo@gmail.com}>. Permission is granted to copy, distribute and/or modify this document under the terms of the GNU Free Documentation License, Version 1.3 or any later version published by the Free Software Foundation; with no Invariant Sections, no Front-Cover Texts, and no Back-Cover Texts.}
  \end{tabularx}

  %\columnbreak

\end{multicols}

\end{document}
